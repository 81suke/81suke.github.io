\documentclass[onecolumn,10pt]{jarticle}
\usepackage[utf8]{inputenc}
\usepackage[top=20truemm,left=20truemm,right=20truemm,bottom=20truemm]{geometry}
\geometry{a4paper}
\usepackage[dvipdfm]{graphicx}
\usepackage[dvipdfmx]{color}
\usepackage{graphicx}
\usepackage{amsmath,amssymb}
\usepackage{fancybox,ascmac}
\usepackage{mathtools}
\usepackage{here}
\usepackage{abstract}
\usepackage{booktabs}
\usepackage{array}
\usepackage{paralist}
\usepackage{verbatim}
\usepackage{subfig}
\usepackage{time}
\usepackage{url}
\usepackage{siunitx}
\usepackage[nottoc,notlof,notlot]{tocbibind}
\usepackage[titles,subfigure]{tocloft}
\newcommand{\red}[1]{\textcolor{red}{#1}}
\renewcommand{\abstractname}{}
\renewcommand{\thesection}{\arabic{section}}
\renewcommand{\thesubsection}{\fbox{\arabic{subsection}}}

\title{数II確認テスト}
\author{}
\date{}
\begin{document}
\maketitle
\begin{itemize}
    \item テキスト、ノートは何も見てはならない。
    \item 問題は3題である。全てに解答すること。
    \item 解答には答えのみでなく途中経過も付すこと。
    \item 解答時間は25分である。
\end{itemize}
\clearpage
\subsection{$直線y=3x+mが円x^2+y^2=4に接するように定数mの値を求めよ。(50点)$}

\clearpage
\subsection{$次の関数の最大値、最小値を求めよ。(50点)$}
\begin{align*}
    y=\sin^2\theta-2\cos\theta-1 (0\leq\theta<2\pi)
\end{align*}

\clearpage
\subsection{$エセンス数\rm{II}を一通り終えた感想を書くこと。なぜ、そのような感想を持ったのか,理由とともに論理的に説明すること。また、どのようにすればよりよい理解になると考えるか,あわせて自らの意見を述べよ。ただ単に,つまらなかった,難しかった,やる気が出なかった,頑張ればよくなる,というような解答には点を与えない。(0点)$}

\end{document}