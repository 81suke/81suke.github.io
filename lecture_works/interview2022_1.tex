\documentclass[onecolumn,10pt]{jarticle}
\usepackage[utf8]{inputenc}
\usepackage[top=19truemm,left=20truemm,right=20truemm,bottom=24truemm]{geometry}
\geometry{a4paper}
\usepackage[dvipdfm]{graphicx}
\usepackage[dvipdfmx]{color}
\usepackage{graphicx}
\usepackage{amsmath,amssymb}
\usepackage{fancybox,ascmac}
\usepackage{mathtools}
\usepackage{here}
\usepackage{array}
\usepackage{paralist}
\usepackage{verbatim}
\usepackage{subfig}
\usepackage{url}
\newcommand{\red}[1]{\textcolor{red}{#1}}
\renewcommand{\abstractname}{}
\renewcommand{\thesection}{\arabic{section}}
\renewcommand{\thesubsubsection}{\fbox{\arabic{subsubsection}}}
\newcommand{\ctext}[1]{\textcircled{\scriptsize #1}}

\begin{document}
\begin{flushright}
    Interview-1(2022)
\end{flushright}

\subsubsection{多項式$3x^2+5xy-2y^2-x-9y-4$は \boxed{\text{(1)}} 次式であり,その定数項は \boxed{\text{(2)}} である.\\
また,方程式$3x^2+5xy-2y^2-x-9y-4=0$は \boxed{\text{(3)}} を表す.}
\begin{enumerate}[(1)]
    \item \ctext{1}8 \ctext{2}4 \ctext{3}2 \ctext{4}1
    \item \ctext{1}4 \ctext{2}3 \ctext{3}-2 \ctext{4}-4
    \item \ctext{1}円 \ctext{2}放物線 \ctext{3}2直線 \ctext{4}点
\end{enumerate}

\subsubsection{関数$f(x)=3x^2-6x+5$の軸は \boxed{\text{(1)}} ,頂点は \boxed{\text{(2)}} である.またこの関数$f(x)$について$f(x)=0$を考えたとき,解は \boxed{\text{(3)}} .}
\begin{enumerate}[(1)]
    \item \ctext{1}$x=1$ \ctext{2}$x=3$ \ctext{3}$y=1$ \ctext{4}$y=3$
    \item \ctext{1}$(1,4)$ \ctext{2}$(1,2)$ \ctext{3}$(3,-4)$ \ctext{4}$(3,-22)$
    \item \ctext{1}異なる2つの実数解をもつ \ctext{2}ただ1つの実数解をもつ \ctext{3}実数解をもたない
\end{enumerate}

\subsubsection{関数$f(x)=-x^2+2tx-t^2+1$について定義域を$1\leq x\leq3$としたときの最小値を考えることとする.このとき,$t\geq$ \boxed{\text{(1)}} のとき最小値 \boxed{\text{(2)}} をとり,$t<$ \boxed{\text{(1)}} のとき最小値 \boxed{\text{(3)}} を得る.}
\begin{enumerate}[(1)]
    \item \ctext{1}0 \ctext{2}1 \ctext{3}2 \ctext{4}3
    \item \ctext{1}$-t^2+1$ \ctext{2}$-t^2+2t$ \ctext{3}$-t^2+4t-3$ \ctext{4}$-t^2+6t-8$
    \item \ctext{1}$-t^2+1$ \ctext{2}$-t^2+2t$ \ctext{3}$-t^2+4t-3$ \ctext{4}$-t^2+6t-8$
\end{enumerate}

\subsubsection{$\sin20^\circ=0.3420, \cos20^\circ=0.9397, \tan20^\circ=0.3640$がわかっているとき,$\sin110^\circ$を求めたい.このとき,$\sin(180^\circ-\theta)=$ \boxed{\text{(1)}} , $\sin(90^\circ-\theta)=$ \boxed{\text{(2)}} であるから,$\sin110^\circ=$ \boxed{\text{(3)}} である.}
\begin{enumerate}[(1)]
    \item \ctext{1}$\sin\theta$ \ctext{2}$-\sin\theta$ \ctext{3}$\cos\theta$ \ctext{4}$\cos\theta$
    \item \ctext{1}$\sin\theta$ \ctext{2}$-\sin\theta$ \ctext{3}$\cos\theta$ \ctext{4}$\cos\theta$
    \item \ctext{1}0.3420 \ctext{2}0.9397 \ctext{3}0.3640
\end{enumerate}

\subsubsection{$\triangle{ABC}$について正弦定理を用いると, \boxed{\text{(1)}} $=2R$である.このとき$R$は \boxed{\text{(2)}} である.正弦定理を用いれば,三角形の辺の比について$a:b:c=$ \boxed{\text{(3)}} であることがわかる.}
\begin{enumerate}[(1)]
    \item \ctext{1}$a\sin A$ \ctext{2}$\cfrac{a}{\cos A}$ \ctext{3}$\cfrac{a}{\sin A}$ \ctext{4}$\cfrac{\sin A}{a}$
    \item \ctext{1}外接円の半径 \ctext{2}外接円の直径 \ctext{3}内接円の半径 \ctext{4}内接円の直径
    \item \ctext{1}$\angle A:\angle B:\angle C$ \ctext{2}$\sin A:\sin B:\sin C$ \ctext{3}$\cos A:\cos B: \cos C$ \ctext{4}$\cfrac{1}{\sin A}:\cfrac{1}{\sin B}:\cfrac{1}{\sin C}$
\end{enumerate}

\newpage

\subsubsection{$2a=2b$は$a=b$であるための \boxed{\text{(1)}} .また,$ac=bc$は$a=b$であるための \boxed{\text{(2)}} .さらに,$\cfrac{a}{c}=\cfrac{b}{c}$は$a=b$であるための \boxed{\text{(3)}} .}
\begin{enumerate}[(1)]
    \item \ctext{1}必要条件であるが十分条件でない \ctext{2}十分条件であるが必要条件でない \ctext{3}必要十分条件である\\\ctext{4}必要条件でも十分条件でもない
    \item \ctext{1}必要条件であるが十分条件でない \ctext{2}十分条件であるが必要条件でない \ctext{3}必要十分条件である\\\ctext{4}必要条件でも十分条件でもない
    \item \ctext{1}必要条件であるが十分条件でない \ctext{2}十分条件であるが必要条件でない \ctext{3}必要十分条件である\\\ctext{4}必要条件でも十分条件でもない
\end{enumerate}

\subsubsection{データが以下のように与えられたとき,中央値は \boxed{\text{(1)}} である.また,第一四分位数は \boxed{\text{(2)}} ,第三四分位数は \boxed{\text{(3)}} である.
\begin{center}
    1, 2, 4, 4, 5, 7, 7, 9, 11, 11
\end{center}
}
\begin{enumerate}[(1)]
    \item \ctext{1}5 \ctext{2}6 \ctext{3}7
    \item \ctext{1}2 \ctext{2}3 \ctext{3}4
    \item \ctext{1}9 \ctext{2}10 \ctext{3}11
\end{enumerate}

\subsubsection{AさんBさんの順でくじを1本ずつひく.くじの中には,はじめに当たりが3本,はずれが7本入っていることがわかっている.Aさんが当たりを引いたときBさんが当たりを引く確率は \boxed{\text{(1)}} であり,Aさんがはずれを引いたときBさんが当たりを引く確率は \boxed{\text{(2)}} である.したがって,Bさんがくじを当てる確率は \boxed{\text{(3)}} .}
\begin{enumerate}[(1)]
    \item \ctext{1}$\cfrac{2}{10}$ \ctext{2}$\cfrac{3}{10}$ \ctext{3}$\cfrac{2}{9}$ \ctext{4}$\cfrac{3}{9}$
    \item \ctext{1}$\cfrac{2}{10}$ \ctext{2}$\cfrac{3}{10}$ \ctext{3}$\cfrac{2}{9}$ \ctext{4}$\cfrac{3}{9}$
    \item \ctext{1}先に引いたAさんより大きい \ctext{2}先に引いたAさんより小さい \ctext{3}先に引いたAさんに等しい
\end{enumerate}

\subsubsection{6個の球を並べる場合を考える.横一列に並べる場合の並べ方は \boxed{\text{(1)}} 通りである.同様に円形に並べる場合は \boxed{\text{(2)}} 通りである.ここで,この6個の球で円形のネックレスをつくる場合は \boxed{\text{(3)}} 通りである.}
\begin{enumerate}[(1)]
    \item \ctext{1}60 \ctext{2}120 \ctext{3}360 \ctext{4}720
    \item \ctext{1}60 \ctext{2}120 \ctext{3}360 \ctext{4}720
    \item \ctext{1}60 \ctext{2}120 \ctext{3}360 \ctext{4}720
\end{enumerate}

\subsubsection{三角形の内心は \boxed{\text{(1)}} の交点であり \boxed{\text{(2)}} .一方で,外心は \boxed{\text{(3)}} の交点であり \boxed{\text{(4)}} .}
\begin{enumerate}[(1)]
    \item \ctext{1}各辺の垂直二等分線 \ctext{2}各頂点から対辺におろした垂線 \ctext{3}それぞれの内角の二等分線\\\ctext{4}それぞれの外角の二等分線
    \item \ctext{1}各辺からの距離が等しい \ctext{2}各頂点からの距離が等しい \ctext{3}重心と一致する
    \item \ctext{1}各辺の垂直二等分線 \ctext{2}各頂点から対辺におろした垂線 \ctext{3}それぞれの内角の二等分線\\\ctext{4}それぞれの外角の二等分線
    \item \ctext{1}各辺からの距離が等しい \ctext{2}各頂点からの距離が等しい \ctext{3}重心と一致する
\end{enumerate}

\subsubsection{$i$を虚数単位とする.このとき,$i=$ \boxed{\text{(1)}} である.これを用いれば,$\sqrt{-2}\times\sqrt{-3}=$ \boxed{\text{(2)}} である.同様に$\cfrac{\sqrt{-2}}{\sqrt{-3}}=$ \boxed{\text{(3)}} となる.}
\begin{enumerate}[(1)]
    \item \ctext{1}$1$ \ctext{2}$-1$ \ctext{3}$\sqrt{-1}$ \ctext{4}$-\sqrt{-1}$
    \item \ctext{1}$\sqrt{6}$ \ctext{2}$-\sqrt{6}$ \ctext{3}$\sqrt{6}i$ \ctext{4}$\sqrt{6i}$
    \item \ctext{1}$\cfrac{\sqrt{6}}{3}$ \ctext{2}$-\cfrac{\sqrt{6}}{3}$ \ctext{3}$\cfrac{\sqrt{6}i}{3}$ \ctext{4}$-\cfrac{\sqrt{6}i}{3}$
\end{enumerate}

\subsubsection{2次方程式の解と係数の関係を用いれば,$2x^2+3x-2=0$の解$\alpha,\beta$について$\alpha+\beta=$ \boxed{\text{(1)}} , $\alpha\beta=$ \boxed{\text{(2)}} が成り立つ.また,$(\alpha-\beta)^2=(\alpha+\beta)^2-4\alpha\beta$であるから,関数$y=2x^2+3x-2$が$x$軸から切り取る線分の長さは \boxed{\text{(3)}} である.}
\begin{enumerate}[(1)]
    \item \ctext{1}$3$ \ctext{2}$-3$ \ctext{3}$\cfrac{3}{2}$ \ctext{4}$-\cfrac{3}{2}$
    \item \ctext{1}$1$ \ctext{2}$-1$ \ctext{3}$2$ \ctext{4}$-2$
    \item \ctext{1}$1$ \ctext{2}$17$ \ctext{3}$\cfrac{25}{4}$ \ctext{4}$\cfrac{5}{2}$
\end{enumerate}

\subsubsection{関数$y=3\sin\left(2\theta+\cfrac{\pi}{4}\right)$について,周期は \boxed{\text{(1)}} で値域は \boxed{\text{(2)}} である.また,この関数は$y=3\sin2\theta$を \boxed{\text{(3)}} だけ平行移動したものである.}
\begin{enumerate}[(1)]
    \item \ctext{1}$\cfrac{\pi}{2}$ \ctext{2}$\pi$ \ctext{3}$2\pi$ \ctext{4}$4\pi$
    \item \ctext{1}$-1\leq y\leq 1$ \ctext{2}$-2\leq y\leq 2$ \ctext{3}$-\cfrac{\pi}{4}\leq y\leq \cfrac{\pi}{4}$ \ctext{4}$-3\leq y\leq 3$
    \item \ctext{1}$x$軸方向に$\cfrac{\pi}{4}$ \ctext{2}$x$軸方向に$-\cfrac{\pi}{4}$ \ctext{3}$y$軸方向に$-\cfrac{\pi}{4}$ \ctext{4}$y$軸方向に$-\cfrac{\pi}{4}$
\end{enumerate}

\subsubsection{円$x^2+y^2-2x-6y+6=0$の中心は \boxed{\text{(1)}} で半径は \boxed{\text{(2)}} である.また,この円に接し点$(-1,1)$を通る直線の方程式は \boxed{\text{(3)}} と求められる.}
\begin{enumerate}[(1)]
    \item \ctext{1}$(1,3)$ \ctext{2}$(-1,3)$ \ctext{3}$(1,-3)$ \ctext{4}$(-1,-3)$
    \item \ctext{1}$1$ \ctext{2}$2$ \ctext{3}$4$ \ctext{4}$6$
    \item \ctext{1}$y=x+2$ \ctext{2}$y=-2x-1$ \ctext{3}$y=-x$ \ctext{4}$y=1$
\end{enumerate}

\subsubsection{$5^{30}$の桁数を求めたい.このとき,$\log_{10}2=0.3010,\log_{10}3=0.4771$がわかっているとき,$\log_{10}5=$ \boxed{\text{(1)}} である.したがって,$\log_{10}5^{30}=$ \boxed{\text{(2)}} より$5^{30}$は \boxed{\text{(3)}} 桁の数であることがわかる.}
\begin{enumerate}[(1)]
    \item \ctext{1}0.1436 \ctext{2}0.7781 \ctext{3}2.096 \ctext{4}3.322
    \item \ctext{1}4.308 \ctext{2}23.34 \ctext{3}62.88 \ctext{4}99.66
    \item \ctext{1}4 \ctext{2}5 \ctext{3}23 \ctext{4}24 \ctext{5}62 \ctext{6}63 \ctext{7}99 \ctext{8}100
\end{enumerate}

\subsubsection{$|\vec{a}|=u,|\vec{b}|=v,\vec{a}\cdot\vec{b}=t$であるとき,$\vec{a},\vec{b}$によってつくられる三角形の面積を求めたい.三角形$\triangle{ABC}$の面積を三角比によって求めるとき,$S=$ \boxed{\text{(1)}} である.また,$\vec{a}\cdot\vec{b}=$ \boxed{\text{(2)}} である.したがって,面積は \boxed{\text{(3)}} である.}
\begin{enumerate}[(1)]
    \item \ctext{1}$ab\sin{C}$ \ctext{2}$ab\cos{C}$ \ctext{3}$\cfrac{1}{2}ab\sin{C}$ \ctext{4}$\cfrac{1}{2}ab\cos{C}$
    \item \ctext{1}$ab\sin{C}$ \ctext{2}$ab\cos{C}$ \ctext{3}$\cfrac{1}{2}ab\sin{C}$ \ctext{4}$\cfrac{1}{2}ab\cos{C}$
    \item \ctext{1}$\cfrac{t}{2}$ \ctext{2}$\cfrac{1}{2}(uv-t)$ \ctext{3}$\cfrac{1}{2}\sqrt{u^2v^2-uvt}$ \ctext{4}$\cfrac{1}{2}\sqrt{u^2v^2-t^2}$
\end{enumerate}

\end{document}